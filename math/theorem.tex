\begin{itemize}
    \item Lucas's Theorem :\\
      For $n, m \in \mathbb{Z}^{*}$ and prime $P$,
      $C(m,n) \mod P$
      %= C(\frac{m}{M},n/M) * C(m\%M,n\%M) mod P
        $= \Pi ( C(m_i,n_i) )$
      where $m_i$ is the $i$-th digit of $m$ in base $P$.
    \item Stirling approximation : \\
      $n!\approx\sqrt{ 2 \pi n}(\frac{n}{e})^{n}e^\frac{1}{12n}$
    \item Stirling Numbers(permutation $|P|=n$ with $k$ cycles): \\
      $S(n,k) = \text{coefficient of }x^k \text{ in } \Pi_{i=0}^{n-1} (x+i)$
    \item Stirling Numbers(Partition $n$ elements into $k$ non-empty set): \\
      $S(n,k) = \frac{1}{k!} \sum\limits_{j=0}^k (-1)^{k-j} {k \choose j} j^n$
    \item Pick's Theorem : $A = i + b/2 - 1$\\
      $A$: Area、 $i$: grid number in the inner、 $b$: grid number on the side
    \item Catalan number : $C_n = {2n \choose n}/(n+1)$\\
      $C^{n+m}_{n}-C^{n+m}_{n+1} = (m+n)! \frac{n-m+1}{n+1}\quad for \quad  n \ge m$\\
      $C_n = \frac{1}{n+1}{2n \choose n} = \frac{(2n)!}{(n+1)!n!}$\\
      $C_0 = 1 \quad  and \quad C_{n+1}= 2(\frac{2n+1}{n+2})C_n$\\
      $C_0 = 1 \quad  and \quad C_{n+1} = \sum_{i=0}^{n} C_iC_{n-i} \quad for \quad  n \ge 0$
    \item Euler Characteristic: \\
      planar graph: $V-E+F-C=1$ \\
      convex polyhedron: $V-E+F=2$ \\
      $V,E,F,C$: number of vertices, edges, faces(regions), and components
    \item Kirchhoff's theorem : \\
      $A_{ii} = deg(i), A_{ij} = (i,j) \in E\ ? -1 : 0$,
      Deleting any one row, one column, and cal the det(A)
    \item Polya' theorem ($c$ is number of color,$m$ is the number of cycle size): \\
      $(\sum_{i=1}^{m}{c^{gcd(i,m)}})/m$
    \item Burnside lemma: \\
      $|X/G| = \frac{1}{|G|}\sum\limits_{g\in G} |X^g|$
    \item 錯排公式 :  ($n$個人中,每個人皆不再原來位置的組合數): \\
      $dp[0]=1;dp[1]=0;$\\
      $dp[i]=(i-1)*(dp[i-1]+dp[i-2])$;
    \item Bell數 (有$n$個人,把他們拆組的方法總數) : \\
      $B_0= 1$\\
      $B_n= \sum_{k=0}^{n} s(n,k)\quad (second-stirling)\\
      B_{n+1}= \sum_{k=0}^{n}{n \choose k} B_k$
    \item Wilson's theorem :\\
      $(p-1)! \equiv -1 (mod \ p)$
    \item Fermat's little theorem :\\
      $a^p \equiv a (mod \ p)$
    \item Euler's totient function:\\
      $ A ^ {B ^ C} mod \ p = pow(A,pow(B,C,p-1)) mod \ p$
    \item 歐拉函數降冪公式:\\
      $A^B \mod C=A^{B \mod \phi(c) + \phi(c)}\mod C$
    \item 6的倍數: \\
     $(a-1)^3 + (a+1)^3 + (-a)^3 + (-a)^3 = 6a$
    \item Standard young tableau (標準楊表): \\
      $\lambda=(\lambda _{1}\geq \cdots \geq \lambda _{k})$,$\sum \lambda _{i} = n$ denoted by $\lambda \vdash n$\\
      $\lambda \vdash n$ 意思為 $\lambda$ 整數拆分 $n$ eg. $n = 10,\lambda = (6,4)$ 此拆分可表示一種楊表形狀。\\
      楊表 : 第$1$列 $\lambda _{1}$ 行 $\cdots$ 第$k$列 $\lambda_{k}$ 行的方格圖。\\
      標準楊表 : 每列從左到右遞增,每行從上到下遞增。 \\
      Let $T$ 為某一 Permutation 跑RSK後的標準楊表,則此Permutation的LDS、LIS長度分別為 $T$ 的列、行數。 
    \item RSK Correspondence: \\
      A permutation is bijective to $(P,Q)$ 一對標準楊表 \\
      $P$ : Permutation 跑 RSK 算法的結果,可為半標準楊表。 \\
      $Q$ : 可用來還原 Permutation (像排列矩陣)。 
    \item Hook length formula (形狀為 $\lambda$ 的標準楊表個數): \\
      $f^{\lambda }={\frac {n!}{\prod h_{\lambda }(i,j)}}$ \\
      $h_{\lambda }(i,j)=$ number of pair $(x,y)$ where $(x = i \lor y = j) \land (x,y) \geq (i,j)$ 且 $(x,y)$ 落在形狀為$\lambda$的表上。

\end{itemize}